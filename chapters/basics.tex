\documentclass[../main]{subfiles}

\begin{document}

\section{Setup}

    \subsection{Vim}

        These commands enable syntax highlighting, line-numbering, and auto indentation, and change tab spacing to 4.

        \monospace{
            \$ vim ~/.vimrc  \\
            syntax on        \\
            set ai           \\
            set number       \\
            set ts=4
        }        

    \subsection{Using the Template}

        Use \code{mkdir /tmp/code/ \&\& cd /tmp/code/ \&\& vim template.cpp}, then type the following:

            \cppfile{basic/template}
        
        Then, initialize the directory as follows:\\
        
        \monospace{
            for x in a b c d e f g h i j k;
            do cp template.cpp \$x.cpp \&\& touch \$x.dat; 
            done
        }

    \subsection{Makefile}
    
        Use \code{vim Makefile}, then enter the following:\\
        
        \monospace{
            CXX=g++                                     \\
            CXXFLAGS=-Wall -static -g -O2 -std=gnu++14  \\
            a: a.cpp                                    \\
            ...                                         \\
            k: k.cpp
        }\\

        This allows you to type \code{make a \&\& ./a < a.dat} to make and run problem A in one go. The compiler flags used are the same ones used by the SCUSA regionals judge in 2017, plus \code{-Wall} to help with debugging.

\end{document}