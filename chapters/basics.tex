\documentclass[../main]{subfiles}

\begin{document}

\section{Template}
    \subsubsection*{The Mother Program}
        Every C++ program should begin with this.
        The \code{\#include} statement will bring in the entire C++11 STL when compiled with \code{g++}, and the strange I/O statements will prevent standard IO from syncing with each other. This will speed up a program that reads from stdout/stdin substantially, at the cost of making \code{scanf} and \code{printf} completely unusable, so use only if you plan on using purely \code{cin} and \code{cout}.
        
        \cppfile{basic/skeleton}

    \subsubsection*{Makefile}
    To compile programs, use the following makefile, which incorporates the exact command used by the judges:
    \begin{cpp}
    main: main.cpp
        g++ -g -lm -lcrypt -O2 -std=c++11 main.cpp -o main
    clean:
        rm main
    \end{cpp}

\subsection{Important Note on Snippets}
    The snippets in this book are \textbf{ALL} written under the assumption that they are being inserted into the skeleton delineated above, i.e. they operate assuming full access to all headers in the C++11 STL using the \code{std} namespace. \textbf{DO NOT USE THEM} without this. You have been warned!

\end{document}