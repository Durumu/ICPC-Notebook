\documentclass[../main]{subfiles}

\begin{document}

\section{Setup}

    \subsection{Vim}

        My preferred lightweight setup file. The macro at the end makes F5 save, compile, and run the program (pretty neat!)

        \monospace{
            \$ vim ~/.vimrc \\
            syntax on       \\
            set ai          \\
            set nu          \\
            set ts=4        \\
            set sw=4        \\
            set is          \\
            set hs          \\
            set bg=dark     \\
            nnoremap <silent> <F5> :w <bar> :make <bar> :!./main < in.dat
        }        

    \subsection{Using the Template}

        Use \code{mkdir /tmp/code/ \&\& cd /tmp/code/ \&\& vim template.cpp}, then type the following:

            \cppfile{basic/template}
        
        Then, initialize the directory as follows:\\
        
        \monospace{
            for x in a b c d e f g h i j k;
            do cp template.cpp \$x.cpp \&\& touch \$x.dat; 
            done
        }

    \subsection{Makefile}
    
        Use \code{vim Makefile}, then enter the following:\\
        
        \monospace{
            CXX=g++                                     \\
            CXXFLAGS=-Wall -static -g -O2 -std=gnu++14  \\
            a: a.cpp                                    \\
            ...                                         \\
            k: k.cpp
        }\\

        This allows you to type \code{make a \&\& ./a < a.dat} to make and run problem A in one go. The compiler flags used are the same ones used by the SCUSA regionals judge in 2017, plus \code{-Wall} to help with debugging.

\section{Choice of Programming Language}

    About $10^6$ operations are doable by Python in 30 seconds, while $10^9$ operations can be done by C++. Therefore, Python should only be used if integers are very big (Python's integers are arbitrarily large), if decimals require arbitrary precision (\code{import decimal} will take care of that.), or for problems where speed is not an issue that involve something like string formatting. The judge guarantees problems to be solvable in C++, but does not guarantee them to be solvable in Python. For this reason, the C++ STL is a far better choice for competitive programming, with  vast majority of code written here is in C++. 

\end{document}