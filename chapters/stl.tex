\documentclass[../main]{subfiles}
    
\begin{document}

\section{Input and Output}

    The bread and butter of every program. This is what you use to read from stdin and write to stdout, in the form of \code{cin} and \code{cout}. I use the headers \code{<iostream>}, \code{sstream}, and \code{<iomanip>} for this purpose. Some people prefer \code{<cstdio>}, which is about a hundred times faster by default, but with the tricks I use in my template file, is not significantly faster, at least not enough to offset the unbelievable headache that is \code{scanf}.\\
    
    \subsection{Output Tricks}

        \code{<iomanip>} is kind enough to provide us with enough functionality for \code{cout} and other output streams to be as flexible as \code{printf}. These are passed directly to \code{cout} or any other output stream (e.g. \code{cout << setw(6)}), and include (but are not limited to):
        
        \begin{itemize}
            \item \code{setbase(int n)}: changes the base that numbers are outputted in to \code{n}, so long as \code{n} is 8, 16, or 10. If you want any other base, you're going to have to write your own code.\footnote{Although base 2 is conspicuously absent from \code{setbase}, there is another solution. The idiomatic way to output an e.g. 32-bit \code{int}'s binary representation is the moderately clunky \code{int x = 53; bitset<32>(x) y; cout << y;}.}. If you want numbers to display their base (e.g. \code{0x1c}) you can pass \code{cout << showbase}, and if you stop wanting that, \code{cout << noshowbase}.
            
            \item \code{setw(int n)}: sets the width of the next singular object passed to \code{cout} to \code{n}. Note that things are right-justified by default; the statement \code{cout << left} will change that, and \code{cout << right} will change it back.
            
            \item \code{setfill(char c)}: changes the fill character (\code{' '} by default).
            
            \item \code{setprecision(int n)}: changes the decimal precision for subsequent floating-point values. By default, this is the number of significant digits which is almost never what you want; pass \code{cout << fixed} first to alleviate this issue.\footnote{If, for some reason, you find yourself wanting to represent floating-point numbers differently, pass \code{cout << scientific} for scientific notation, \code{hexfloat} for hexadecimal, or \code{defaultfloat} for the default float notation.}

            \item \code{flush}: flushes the output buffer. I define \code{endl} to be \code{"{\textbackslash}n"}, trading its flushing functionality for speed, so if you need flushing for some reason, here it is.
        \end{itemize}

    \subsection{Input Tricks}

        \code{cin} uses the same \code{<iomanip>} flags as \code{cout}, but with a few twists. For one, using \code{setbase(0)}, which is the default value, allows \code{cin} to infer the base of integers based on what precedes them (e.g. \code{0x} for hex). Another useful flag is \code{ws}; passing this to \code{cin} eats the whitespace.\\

        Another useful trick for input is \code{getline(cin, s)} where \code{s} is some string. This puts the entire line, excluding the trailing newline, into \code{s}, which can then be checked to see if it is empty. A \code{stringstream} can then be constructed around \code{s} to process it; but beware as this is highly inefficient, so should only be used if there isn't a lot of input. If there is a lot of input, there is a workaround, although it gets messy: use \code{cin} to extract the normal way, then \code{getline(cin, dump)} for some random junk string, then \code{cin.peek() == \"{\textbackslash}n\"}. Not exactly elegant, but works fine.\\

        Either \code{getline(cin, s)} or \code{cin >> x} can be wrapped in a \code{while} loop's condition, which will terminate at EOF.

\section{Algorithms}



\section{Containers}

\end{document}    