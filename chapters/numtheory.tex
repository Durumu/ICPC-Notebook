\documentclass[../main]{subfiles}
    
\begin{document}

Unless explicitly stated otherwise, numbers are assumed to belong to $\mathbb{N}$.

\section{Definitions}

    \begin{itemize}
    \item
        A number $p$ is \textit{prime} if its only divisors are 1 and itself.
    \item 
        Two numbers $a$ and $b$ are \textit{coprime} if $gcd(a, b) = 1$. 
    \item
        A \textit{perfect number} $n$ is equal to the sum of its proper positive divisors (i.e. the divisors less than $n$). All known perfect numbers are even, and of the form $q(q+1)/2$ where $q$ is a \textit{Mersenne prime} (a prime of the form $2^p - 1$ for some prime $p$).
    \end{itemize}

\section{Primes}

    \subsection{Primes Library}

        \cppfile{numtheory/primes}

    \subsection{Sieve of Eratosthenes}

    The \textbf{sieve of Eratosthenes} is a basic prime sieve. It can find all primes up to $n$ in $O(n\log\log{n})$.\\

    \subsection{Factorization Sieve}

    There is also a linear-time sieve, suggested by 

    \subsection{Euler's Totient Function}

    \textbf{Euler's totient function} $\phi(n)$ for some $n$ is defined as the number of integers less than $n$ that $n$ is relatively prime with, i.e. whose GCD with $n$ is 1. Computing this in $O(\sqrt{n})$ is easy, and the following solution can easily be optimized for many $O(\log n)$ queries.\\

    $\phi(n)$ has many interesting properties, but the most famous and useful is the fact that $a^{\phi(n)} \equiv 1 \pmod{n}$ for coprime $a$, $n$.

    \subsubsection*{The Prime Number Theorem}
    
        $\pi(x)$, the number of primes less than some x, grows 
        at almost exactly the same rate as $\frac{x}{\log{x}-1}$.

\section{Modular Arithmetic}

    \subsection{GCD and LCM}

        \cppfile{numtheory/gcd}

    \subsection{Euclid Codebase}

        \cppfile{numtheory/euclid}

    \subsection{Modular Inverse}

    The \textbf{extended Euclidean algorithm} retains the sublinear time complexity of Euclid's simple algorithm, and for almost no extra cost finds integer coefficients $x$, $y$ for the equation $ax + by = gcd(a, b)$.
    

    This equation is useful because it allows us to find the \textbf{modular inverse} of two numbers $a$ and $m$, i.e. the number $x$ such that $ax \equiv 1 \pmod{m}$, which can then be used for other powerful results, like the Chinese remainder theorem. The reason this works is that a modular inverse for $a \pmod{m}$ is only possible if $a$ and $m$ are coprime, i.e. $gcd(a, m) = 1$. Thus, if $ax + my = 1$, $ax - 1 = (-y)m$, and thus $ax \equiv 1 \pmod{m}$.


    \subsection{Chinese Remainder Theorem}

    The \textbf{Chinese remainder theorem} enables us to solve for the lowest possible $x$ satisfying a system of equations of the form $x\equiv a \pmod {d}$, for two equally-sized vectors $a$ and $d$, provided each pair of elements in $d$ is coprime. This is possible using Euclid's extended algorithm to find the inverse GCD.

    \subsection{Legendre's Formula}

    Given some prime number $p$ and some $n$, the largest number $x$ such that $n!$ is evenly divisible by $p^x$ is $\sum\floor*{\frac{n}{p^i}}$.

\end{document}    
