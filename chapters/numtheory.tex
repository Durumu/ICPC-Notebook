\documentclass[../main]{subfiles}
    
\begin{document}

\section{Modular Arithmetic}



\subsection{GCD and LCM}

\cppfile{numtheory/gcd}

\subsection{Chinese Remainder Theorem}

\subsection{Legendre's Formula}

Given a prime number $p$ and natural number $n$, the largest natural number $x$ such that $n!$ is evenly divisible by $p^x$ is $\sum_{i}\floor*{\frac{n}{p^i}}$.

\section{Primes}

\subsection{Prime Facts}

\subsubsection*{The Prime Number Theorem}

$\pi(x)$, the number of primes less than some $x\in\mathbb{N}$, grows 
at almost exactly the same rate as $frac{x}{\log{x}-1}$.

\subsubsection*

\subsection{The Sieve of Eratosthenes}

The sieve of Eratosthenes is an extremely efficient and useful algorithm, finding all primes up to $n$ in $O(n\log\log{n})$.

\cppfile{numtheory/primesieve}

Generally, the obvious use is sufficient. However, sometimes, e.g. when checking a few very large numbers for primality, it is best to create a prime sieve of size $\sqrt{n}$, where $n$ is the largest number, then use the sieve to create a sorted vector of all primes up to $\sqrt{n}$ and test each of these individually. This technique is also helpful for efficient factorization of large numbers (in $O(log n)$) through continuous division.

\cppfile{numtheory/primes_to}

\section 

\end{document}    